\documentclass{customcv}

\hypersetup{
    pdftitle={Peter Kopáč - Frontend Developer - Životopis},
    pdfauthor={Peter Kopáč},
    pdfsubject={Frontend Developer (React) - <tvoj-email>},
    pdfkeywords={React, Node.js, Nextjs, JavaScript, TypeScript, Frontend, GitHub: <tvoj-github>, Portfolio: <tvoj-web>}
}

\begin{document}

\makecvheader{Peter Kopáč}{<datum narodenia>}
    {<adresa>}
    {\href{https://linkedin.com/in/<Inkedin>}{linkedin.com/in/<Inkedin>}}
    {\href{mailto:<email>}{<email>}}
    {\href{https://github.com/<github>}{github.com/<github>}}
    {<telefon>}
    {\href{https://<webstreanka>}{<webstreanka>}}

Som Software Developer zameraný na Frontend alebo Fullstack vývoj. Zakladám si na silných technických základoch, ako sú algoritmy, dátové štruktúry, design patterns, Clean Code a SOLID princípy. Tie mi umožňujú rýchlo sa adaptovať, učiť sa nové technológie a písať čistý, optimalizovaný kód.

Pri riešení problémov rád využívam kreativitu a schopnosť myslieť do hĺbky. Aj vďaka dyslexii dokážem k zadaniam pristupovať z iného pohľadu a hľadať inovatívne riešenia.

\section{Skúsenosti a Projekty}

\cvexperience{Software Developer / Consultant}{Projektová spolupráca}{07.2019 - 04.2024}{React, Next.js, NestJS, Node.js, GraphQL, Unity 3D, PostgreSQL, Docker}
Remote. Spolupráca so startupmi a technologickými firmami na vývoji projektov na zelenej lúke v oblastiach Health Care, Management a herný priemysel.Väčšina kontraktov bola viazaná na prenájom expertnej kapacity, na konkrétne technické ciele alebo uvedenie produktov do produkcie.. Nižšie uvádzam vybrané projekty:

\begin{itemize}
    \item \cvsubproject{Amabels \duration{2 mesiace}}{Vývoj nového dizajnu pre autorizáciu a authetifikáciu v Angular a Node.js pod microservices architektúrou.}
    
    \item \cvsubproject{01People \& Hilbi \duration{3 mesiace}}{Spolupráca na vývoji rezervačného systému pre lekárov v technológii React. Úpravy na landing pages v Next.js pre Premedix Clinic, Premedix Academy a Caresee.}
     
    \item \cvsubproject{Cognitive Talent Solutions \duration{6 mesiacov}}{Kompletný vývoj platformy na analýzu vnútrofiremných vzťahov. Navrhol a implementoval som frontend v Reacte a škálovateľný backend v Node.js/GraphQL.}
     
    \item \cvsubproject{Afinode \duration{1 rok a 9 mesiacov}}{Fullstack vývoj mobilnej hry Baron Bite. Ide solo vývoj od návrhnu, po implmentáciu až po release.}
\end{itemize}

\cvexperience{Software Developer}{QICS (FORTES)}{04.2024 - 07.2024}{AngularJS, Angular, Microfrontends, ASP.NET}
Žilina. Migrácia legacy systému z AngularJS na moderný Angular v rámci mikrofrontendovej architektúry. Aktívna účasť na návrhu technických rozhodnutí v rámci tímu. Spolupráca v agilnom tíme (Scrum), pravidelné prezentácie progresu a spolurozhodovanie o ďalšom smerovaní projektu.

\cvexperience{Trainee React Developer}{Unicorn}{05.2019 - 07.2019}{React}
Ostrava. Programoval som špecializované widgety v Reacte pre projekt Lancelot v oblasti energetiky.
Prispieval som k tvorbe ukážkových projektov v rámci korporátneho SaaS prostredia.
Aktívne som sa zúčastňoval dennej komunikácie a odborného code-review v Scrum tíme.

\cvexperience{.NET Developer}{Sigp (CODERAMA)}{02.2019 - 04.2019}{C\#, Oracle, PL/SQL, WinForms}
Vývoj enterprise ERP systému pre Úrad regulácie hazardných hier v rámci špecializovaného outsourcingového tímu pre klienta Asseco Solution.
Moja práca zahŕňala implementáciu nových modulov v C\# a správu komplexnej databázovej vrstvy v Oracle (PL/SQL) a koordináciu so seniornými vývojármi a analytikmi.

\newpage
\vspace*{1em}

\cvexperience{Junior Chatbot Developer}{Aviget}{09.2018 - 12.2018}{Angular, C\#, ASP.NET Core, MSSQL}
Zodpovednosť za architektonický návrh a implementáciu základného aplikačného rámca (boilerplate) pre Greenfield SaaS platformu zameranú na chatbot riešenia v leteckom priemysle. V rámci malého tímu s využitím XP Programmingu som sa primárne zameriaval na vývoj frontendovej časti v Angular a integráciu kritických modulov pre autorizáciu a autentifikáciu používateľov.

\section{Vzdelanie}
\textbf{Vysoká škola báňská – Technická univerzita Ostrava} | 2015 - 2018 (nedokončené) \\
Fakulta Elektrotechniky a Informatiky. Študijný odbor: Informatika a výpočtová technika (denné štúdium). \vspace{0.4em} \\
\textbf{Spojená Škola – Kysucké Nové Mesto} | 2011 - 2014 \\
Študijný odbor: Elektrotechnika. Názov maturitného projektu: Učebnica programovania na Android.

\section{Aktivity}
\cvactivity{\textbf{Game Jam Fest Anča Žilina} (2016–2018). Vývoj prototypov hier za 48 hodín (Unity 3D, C\#, Cinema 4D, Blender) na témy ako „V jednote je sila“ alebo „Sám proti sebe“.}
\cvactivity{\textbf{Global Game Jam} (2018). Vývoj prototypov hier za 48 hodín (Unity 3D, C\#, Cinema 4D, Blender) na tému Prenos}
\cvactivity{\textbf{Startup Weekend Žilina} (2017, 2018). Postavenie základov inovatívnej firmy za 54 hodín.}
\cvactivity{\textbf{Hackathony Kovork Ostrava} (2015–2018). Účasť na 6 podujatiach (React, Node.js, Unity 3D, Firebase). Vývoj sociálnych sietí, vizualizácií dát a hier s prvkami AI v limite 30 hodín.}

\section{Znalosti}
\cvskill{Cudzie jazyky}{Angličtina (B1)}
\cvskill{Programovanie}{JavaScript, TypeScript, C\#,}
\cvskill{Frontend}{React, Next.js, Angular, Redux, Mobx, React Query, NgRx, RxJS, HTML, Tailwind, SASS, \\ Styled-components  , (Ant Design, Material, Bootstrap...)}
\cvskill{Backend}{REST API, GraphQL, Node.js, Express.js, Nest.js, Socket.io, ASP.NET Core, PostgreSQL, \\ MongoDB, Redis}
\cvskill{Testing tools}{Postman, Jest, React Testing Library, Cypress, Playwright}
\cvskill{Cloud, DevOps}{Docker, Firebase, Linux, AWS, Grafana}
\cvskill{Game dev}{Unity 3D, Game Design, OpenGL, Custom shader development}
\cvskill{Ostatné}{Jira, Git, SOLID principles, Clean Code, Event-driven, Serverless, Modular Monolith, Headless}

\section{Certifikácie}
CCNA Exploration: Network Fundamentals (2014) \\
CCNA Routing and Switching: Routing and Switching Essentials (2015) \\
{\href{https://app.pluralsight.com/profile/peter-kopac}{app.pluralsight.com/profile/peter-kopac}}

\end{document}
\documentclass{customcv}

\hypersetup{
    pdftitle={Peter Kopáč - Frontend Developer - Životopis},
    pdfauthor={Peter Kopáč},
    pdfsubject={Frontend Developer (React)},
    pdfkeywords={React, JavaScript, TypeScript, Frontend, Software Developer, Web Development}
}

\begin{document}

\makecvheader{Peter Kopáč}{<datum narodenia>}
    {<adresa>}
    {\href{https://linkedin.com/in/<Inkedin>}{linkedin.com/in/<Inkedin>}}
    {\href{mailto:<email>}{<email>}}
    {\href{https://github.com/<github>}{github.com/<github>}}
    {<telefon>}
    {\href{https://<webstreanka>}{<webstreanka>}}

Som Frontend Developer (React) so 7-ročnou praxou v komerčnom vývoji. Reactu sa venujem od roku 2017 a programovaniu celkovo od roku 2013. Okrem Reactu som sa komerčne stretol tiež s Angular, .NET, Node.js a Unity 3D. 

Vynikám analytickým myslením, znalosťou algoritmov, dátových štruktúr a design patternov, čo mi umožňuje rýchlu adaptáciu na nové technológie. Dbám na čistý a optimalizovaný kód a rád prejavujem kreativitu pri riešení problémov. 

Napriek dysgrafii a dyslexii efektívne plním svoje úlohy a pri riešení problémov dokážem myslieť do hĺbky a prichádzať s riešeniami z iného pohľadu.

\section{Skúsenosti a Projekty}

\cvexperience{Software Developer}{QICS (FORTES)}{04.2024 -- 07.2024}{AngularJS, Angular, Microfrontends, ASP.NET}
Žilina, zmluva na dobu určitú, full-time. Spolupráca v Scrum tíme (do 10 ľudí) zloženom zo Scrum Mastera, Product Ownera, developerov a testerov. Migrácia projektu z AngularJS na novšiu verziu Angular pod mikrofrontendovou architektúrou. Účasť na denných stand-upoch, pravidelných plánovaniach práce a code-review.

\cvexperience{Software Developer}{Externe projekty}{07.2019 -- 04.2024}{Node.js, Nest.js, Express.js, React, Angular}
Remote. Drobné práce na klientskych projektoch.
\begin{itemize}
    \item \cvsubproject{Amabels}{Rezervačný systém pre Amabels (Angular, Node.js, MongoDB, Microservices). Vývoj informačného systému na tvorbu rezervácií podľa požiadaviek klienta.}
    \item \cvsubproject{01People \& Hilbi}{Spolupráca na vývoji rezervačného systému pre lekárov v technológiách React a Next.js. Úpravy na landing pages v Next.js pre Premedix Clinic, Premedix Academy a Caresee.}
    \item \cvsubproject{Afinode, Baron Bite}{Mobilná hra (Unity 3D, Firebase, Node.js). Komunikácia s klientom, zber a spracovanie požiadaviek, implementácia herných mechaník a oprava chýb v backendovom systéme a v mobilnej hre.}
\end{itemize}

\cvexperience{Software Developer}{Cognitive Talent Solutions}{07.2019 -- 04.2024}{React, Node.js, GraphQL, PostgreSQL, MongoDB}
Vývoj Greenfield projektu v startupovej firme pre analyzovanie vzťahov zamestnancov vo firme. Spolupráca v malom tíme (2x Developer, Analytik) pod Kanban. Implementácia nových a optimalizácia existujúcich funkcií na frontendovej aplikácii v React a na backendovej strane v Node.js.

\cvexperience{Trainee React Developer}{Unicorn}{05.2019 -- 07.2019}{React}
Ostrava, TPP, full-time. Programoval som špecializované widgety v Reacte pre projekt Lancelot v oblasti energetiky. Prispieval som k tvorbe ukážkových projektov v rámci korporátneho SaaS prostredia. Aktívne som sa zúčastňoval dennej komunikácie a odborného code-review v Scrum tíme.

\newpage
\vspace*{1em}

\cvexperience{Junior .NET Developer}{Sigp (CODERAMA)}{02.2019 -- 04.2019}{C\#, Oracle, PL/SQL, WinForms}
Žilina, TPP, full-time. Poskytovateľ IT outsourcingu. Tím (3 členovia) so senior developerom a analytikom z Asseco Solution. Vývoj ERP systému pre Úrad Regulácie Hazardných Hier.

\cvexperience{Junior Chatbot Developer}{Aviget}{09.2018 -- 12.2018}{Angular, C\#, ASP.NET Core, MSSQL}
Remote, zmluva na dobu určitú, full-time. Startup, Greenfield SaaS pre letecké spoločnosti. Malý tím 2x developer a analytik, XP Programming. Vývoj prevažne frontendovej časti v Angular.

\section{Vzdelanie}
\textbf{Vysoká škola báňská – Technická univerzita Ostrava} | 2015 -- 2018 (nedokončené) \\
Fakulta Elektrotechniky a Informatiky. Študijný odbor: Informatika a výpočtová technika (denné štúdium). \vspace{0.4em} \\
\textbf{Spojená Škola – Kysucké Nové Mesto} | 2011 -- 2014 \\
Študijný odbor: Elektrotechnika. Názov maturitného projektu: Učebnica programovania na Android.

\section{Aktivity}
\cvactivity{\textbf{Game Jam Fest Anča Žilina / Global Game Jam} (2016–2018). Vývoj prototypov hier za 48 hodín (Unity 3D, C\#, Cinema 4D, Blender) na témy ako „V jednote je síla“ alebo „Prenos“.}
\cvactivity{\textbf{Startup Weekend Žilina} (2017, 2018). Postavenie základov inovatívnej firmy za 54 hodín.}
\cvactivity{\textbf{Hackathony Kovork Ostrava} (2015–2018). Účasť na 6 podujatiach (React, Node.js, Unity, Firebase). Vývoj sociálnych sietí, vizualizácií dát a hier s prvkami AI v limite 30 hodín.}

\section{Znalosti}
\cvskill{Cudzie jazyky}{Angličtina (B1)}
\cvskill{Programovanie}{JavaScript, TypeScript, C\#, C/C++}
\cvskill{Front-End}{React, Next.js, Angular, Redux, Mobx, React Query, NgRx, RxJS, HTML, \\ CSS/Tailwind/SASS/Styled Component, Design Systems (Ant Design, Material, Bootstrap...)}
\cvskill{Back-End}{REST API, GraphQL, Node.js, Express.js, Nest.js, Socket.io, ASP.NET Core, PostgreSQL, \\ MongoDB, Redis}
\cvskill{Testing tools}{Postman, Jest, React Testing Library, CypressS, Playwright}
\cvskill{Cloud, DevOps}{Docker, Firebase, Linux, Bash, PowerShell, AWS, Grafana}
\cvskill{Game dev}{Unity 3D, Game AI, Blender, Game Design, OpenGL, GLSL Shadery}
\cvskill{Ostatné}{Jira, Git, SOLID principles, Clean Code, Event-driven, Serverless, CQRS}

\section{Certifikácie}
CCNA Exploration: Network Fundamentals (2014) \\
CCNA Routing and Switching: Routing and Switching Essentials (2015)

\end{document}